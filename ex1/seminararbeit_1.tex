\documentclass{article}

\usepackage{Sweave}
\begin{document}
\Sconcordance{concordance:seminararbeit_1.tex:seminararbeit_1.Rnw:%
1 5 1 1 0 9 1 1 3 2 0 1 1 5 0 1 3 1 0 1 1 5 0 1 3 1 0 1 1 5 0 1 3 1 0 1 %
1 5 0 1 3 1 0 1 1 5 0 1 3 1 0 1 1 6 0 1 2 24 1 1 3 5 0 1 2 2 1 1 3 5 0 %
1 2 8 1 1 3 2 0 1 1 3 0 1 2 2 1 1 3 8 0 1 2 2 1 1 3 8 0 1 2 8 0 1 2 5 1}



\section{Part I}

\subsection{}

R knows six different vector types, namely: logical, integer, real, complex, character (string) and raw. To give some examples for every type:

\begin{Schunk}
\begin{Sinput}
> # define logical object
> log <- TRUE
> is.logical(log)
\end{Sinput}
\begin{Soutput}
[1] TRUE
\end{Soutput}
\begin{Sinput}
> # define integer object
> int <- 1:5
> is.integer(int)
\end{Sinput}
\begin{Soutput}
[1] TRUE
\end{Soutput}
\begin{Sinput}
> # define real (numeric double) object
> real <- 2.5
> is.double(real)
\end{Sinput}
\begin{Soutput}
[1] TRUE
\end{Soutput}
\begin{Sinput}
> # define complex object
> comp <- 1+2i
> is.complex(comp)
\end{Sinput}
\begin{Soutput}
[1] TRUE
\end{Soutput}
\begin{Sinput}
> # define character (string) object
> char <- "a"
> is.character(char)
\end{Sinput}
\begin{Soutput}
[1] TRUE
\end{Soutput}
\begin{Sinput}
> # define raw object
> rawd <- as.raw(22) # corresponds to 16
> is.raw(rawd)
\end{Sinput}
\begin{Soutput}
[1] TRUE
\end{Soutput}
\end{Schunk}

\subsection{}

Difference between generic and numeric vector:

\begin{itemize}
\item An \emph{atomic} vector contains only one single ``atomic'' data type in all entries. An example would be a vector which contains only integers.
\item A \emph{generic} vector (like a \texttt{list}) can contain different types of data. An example would be a vector which contains characters and numbers.
\end{itemize}

\subsection{}

To explain: \emph{A data frame is a list, but not evey list is a data frame.}

\begin{itemize}
\item A \texttt{list} is an object containing collections of objects. The types of the entries inside of the list can be different. It is for example allowed that a \texttt{list} contains a vector of real values (doubles) and a vector of characters. The length of the containing vectors can be \textbf{different}.
\item A \texttt{data frame} is an object containing colletions of objects. The types of the entries inside of the list can be different. The length of the vectors have to be \textbf{the same}. The \texttt{data frame} has a matrix-like structure.
\end{itemize}

\texttt{list} and \texttt{data frame} are very similar, but the \texttt{data frame} has one more restriction (same length of all vectors). That's why a \texttt{data frame} is always a list, but a \texttt{list} is not always a \texttt{data frame}.

\section{Part II}



\end{document}
